\documentclass[a4paper,11pt]{article}
\usepackage{latexsym}
\usepackage[italian]{babel}
\usepackage[utf8]{inputenc}  
\usepackage{pdfsync}
\usepackage{moreverb}
\usepackage{listings}
\author{Alessio Caiazza, Cosimo Cecchi}
\title{CaptureMJPEG: a MotionJPEG library for Procesing}
\frenchspacing
\begin{document}
\maketitle

\newcommand{\reffigura}[1]{
  Figura \ref{#1}
}

\begin{abstract}
CaptureMJPEG è una libreria per
Processing\footnote{http://processing.org} che consente di gestire uno
stream motion-jpeg come input video.\\
La libreria è in grado di acquisire lo stream tramite i protocolli
\mbox{HTTP/HTTPS} e dispone di alcune classi di aiuto per la generazione di
URL per le videocamere di rete AXIS e Sony.  
\end{abstract}
\tableofcontents


\section{Introduzione}
\label{sec:introduzione}
%Cosa è stato fatto, come e con quali obiettivi
%Presentazione del progetto e del sito
\section{Analisi}
\label{sec:analisi}
%Studio della libreria eseguito con il codice di prova
L'analisi è stata svolta applicando un filtro \texttt{blur} allo
stream ottenuto con CaptureMJPEG e con la videocamera locale
utilizzando la libreria Capture\footnote{la libreria Capture è fornita
  in bundle con Processing.}.\\
I sorgenti utilizzati sono quelli in \reffigura{fig:micc_blur} per
CaptureMJPEG e in \reffigura{fig:capture_blur} per Capture.
Sono stati misurati l'utilizzo di memoria e di CPU al variare delle
dimensioni del filmato e del framerate richiesto allo sketch.
\begin{figure}
  \centering
\lstinputlisting[language=Java,numbers=left,frame=shadowbox]{sources/micc_blur.pde}  
  \caption{Sorgente di test CaptureMJPEG}
  \label{fig:micc_blur}
\end{figure}
\begin{figure}
  \centering
  \lstinputlisting[language=Java,numbers=left,frame=shadowbox]{sources/capture_blur.pde}
  \caption{Sorgente di test Capture}
  \label{fig:capture_blur}
\end{figure} 

%qui andrebbero messi i risultati

\section{Manuale}
\label{sec:manuale}
Guida all'installazione ed utilizzo di CaptureMJPEG
\subsection{Installazione}
\label{sec:installazione}
%scopiazzare dal sito e tradurre in italiano
\subsection{Guida all'utilizzo}
\label{sec:guida}
%inserire un po' di esempi e spiegare le funzioni utilizzabili


\section{Sviluppo}
\label{sec:sviluppo}
Come continuare lo sviluppo
\subsection{Ottenere i sorgenti}
\label{sec:sorgenti}
Prima di scaricare i sorgenti è necessario installare
Mercurial\footnote{Mercurial può essere scaricato dal sito
 http://www.selenic.com/mercurial/}, 
per la gestione dei sorgenti ed Ant\footnote{Ant può essere scaricato
dal sito http://ant.apache.org}, 
per la gestione della compilazione.

Per ottenere i sorgenti eseguire la clonazione del repository
mercurial disponibile all'indirizzo
\texttt{http://dev.abisso.org/capturemjpeg} 
dopodiché creare una copia del file \texttt{user\_pref.xml.template}
con nome \texttt{user\_pref.xml}.

Il file contiene la configurazione di ant per il progetto, tutti i
valori di default vanno bene ad eccezione della ``property''
\texttt{processing-core} che deve essere corretta con la path completa
al file \texttt{core.jar} incluso nella propria installazione di Processing.
\begin{verbatim}
<property name="processing-core" 
    value="C:\Programmi\processing-0135-expert\lib\core.jar" />
\end{verbatim}

A questo punto è necessario eseguire il dowload delle librerie incluse
in CaptureMJPEG eseguendo il comando:
\begin{verbatim}
   ant download_deps
\end{verbatim}

Quindi è possibile generare l'intera cartella di installazione con
il comando:
\begin{verbatim}
   ant deploy 
\end{verbatim}
\begin{figure}
  \centering
\begin{boxedverbatim}

 hg clone http://dev.abisso.org/capturemjpeg capturemjpeg   
 cd capturemjpeg
 cp user_pref.xml.template user_pref.xml

\end{boxedverbatim}  
  \caption{Come ottenere i sorgenti da terminale}
  \label{fig:clone}
\end{figure}

\subsection{Classi utilizzate}
\label{sec:classi}
Forniamo ora una descrizione sommaria delle classi sviluppate per la
libreria, per una trattazione più approfondita si rimanda alla
documentazione JavaDoc disponibile online all'indirizzo 
http://capturemjpeg.lilik.it/doc/

%inserire il diagramma delle classi UML e aggiungere qualche commento




\end{document}
